\documentclass{beamer}
\usepackage[utf8]{inputenc}
\usepackage{hyperref}
\usepackage{multicol}
\usepackage{hyperref}
\usepackage{verbatim}

\inputencoding{utf8}

\mode<presentation> {
    \usetheme{Madrid}
}

\usepackage{graphicx}
\usepackage{booktabs}

\title{Soluciones y Tests Unitarios}
\author{Ernesto Rodriguez}
\institute{
    Universidad del Itsmo \\
    \medskip \textit{erodriguez@unis.edu.gt}
}

\date[\today]{}

\begin{document}

\begin{frame}
\titlepage
\end{frame}

\begin{frame}
    \frametitle{Proyect}
    \begin{itemize}
        \item{Conjunto de archivos fuente que se compilan en conjunto para producir un archivo objeto.}
        \item{El archivo objeto es un archivo binario.}
        \item{El archivo binario puede ser una biblioteca o un ejecutable.}
        \item{Cada proyecto deberia producir solamente un archivo objeto.}
        \item{Un proyecto se crea mediante: \texttt{dotnet new [tipo]}}
    \end{itemize}
\end{frame}

\begin{frame}
    \frametitle{Referencia}
    \begin{itemize}
        \item{A menudo un archivo objeto necesita de otro archivo objeto para ejecutarse.}
        \item{Para ello, un archivo objeto debe hacer referencia a otro archivo objeto.}
        \item{El framework, .Net permite que un proyecto haga referencia a otro proyecto,
            lo cual le permite utilizar el archivo objeto de ese proyecto durante
            su ejecuci\'on.}
        \item{Para hacer hacer referencia a otro proyecto se utiliza el comando:
            \texttt{dotnet add reference [camino/a/referencia.csproj]}}
    \end{itemize}
\end{frame}

\begin{frame}
    \frametitle{Soluci\'on}
    {\bf Soluci\'on:} Conjunto de proyectos que dependen entre ellos
    con el proposito de crear un programa complejo.
    \begin{itemize}
        \item{Es importante dividir un programa en varios proyectos,
        esto permite que el programa sea modular.}
        \item{{\bf Principio de unica responsabilidad:} Cada proyecto dentro de la soluci\'on debe cumplir
        solamente una funci\'on.}
        \item{El dies\~no modular permite utilizar componentes de de un
        programa para construir otros programas.}
        \item{Una soluci\'on se crea mediante el comando: \texttt{dotnet new sln}}
    \end{itemize}
\end{frame}

\begin{frame}
    \frametitle{Testing}
    \begin{itemize}
        \item{El compilador realiza una verificaci\'on basica
        de la exactitud de un programa.}
        \item{Sin embargo, esa validaci\'on muchas veces es
        demasiado superficial.}
        \item{Es posible realizar verificaciones adicionales mediante
        otros programas.}
        \item{A esto se le conoce como \emph{testing}}
        \item{Existen varios tipos de testing: unitarios, integraci\'on,
        aceptaci\'on.}
        \item{En este curso solamente estudiaremos los tests unitarios.}
    \end{itemize}
\end{frame}

\begin{frame}
    \frametitle{Pruebas unitarias}
    \begin{itemize}
        \item{Verifican que un proyecto individualmente cumple con las
        funciones que debe llevar a cabo.}
        \item{Deben ser simples y tener un proposito definido.}
        \item{Por lo general se dividen en tres partes:
            \begin{enumerate}
                \item{Preparaci\'on}
                \item{Acto}
                \item{Validaci\'on}
            \end{enumerate}
        }
        \item{Por lo general, cada prueba unitaria se enfoca solamente
        en una o pocas clases.}
        \item{.Net core permite crear un proyecto exclusivo de pruebas
        unitarias mediante el comando: \texttt{dotnet new xunit}}
    \end{itemize}
    {\bf Importante:} Es buena practica colocar las purebas unitarias en
    un proyecto separado al proyecto que esta siendo verificado debido
    a que el codigo de una prueba unitaria no afecta el funcionamiento
    del proyecto principal.
\end{frame}

\end{document}