\documentclass{beamer}
\usepackage[utf8]{inputenc}
\usepackage{hyperref}
\usepackage{multicol}
\usepackage{hyperref}
\usepackage{verbatim}

\inputencoding{utf8}

\mode<presentation> {
    \usetheme{Madrid}
}

\usepackage{graphicx}
\usepackage{booktabs}

\title[Git]{Recursi\'on}
\author{Ernesto Rodriguez}
\institute{
    Universidad del Itsmo \\
    \medskip \textit{erodriguez@unis.edu.gt}
}

\date[\today]{}

\begin{document}

\begin{frame}
\titlepage
\end{frame}

\begin{frame}
    \frametitle{Motivaci\'on}
    \begin{itemize}
        \item{A menudo, una funci\'on se puede definir en terminos de ella misma}
        \item{Cada iteraci\'on de la funci\'on ejecuta una versi\'on simplificada de la misma}
        \item{Dichas implementaciones permiten mayor facilidad para razonar sobre algoritmos}
        \item{El calculo-$\lambda$ original, solamente permitia hacer ciclos mediante
        recursi\'on.}
        \item{Es la versi\'on programatica de la inducci\'on matematica}
    \end{itemize}
\end{frame}

\begin{frame}
    \frametitle{Recursi\'on: Idea}
    \begin{itemize}
        \item{Una funci\'on se define en dos partes:}
        \item{{\bf Caso base:}
            \begin{itemize}
                \item{Es el caso que se considera cuando la funci\'on
                es llamada con los parametros m\'as sencillos. Tambi\'en llamado caso trivial}
                \item{Pueden haber varias condiciones o valores de entrada a la funci\'on para
                las cuales es aplicable el caso base. Ej. suma de numeros unitarios.}
            \end{itemize}
        }
        \item{{\bf Caso recursivo:}
            \begin{itemize}
                \item{Es el caso que considera todos parametros que no cumplen los
                criterios del caso base.}
                \item{El objetivo de este caso es simplificar los parametros y volver
                a ejecutar la funci\'on con los parametros simplificados.}
                \item{Este caso puede ejecutar la misma funci\'on, una o varias veces,
                directa o indirectamente.}
                \item{Cuando una funci\'on (o metodo) se llama a si mismo se conoce
                como \emph{llamada recursiva}}
            \end{itemize}
        }

    \end{itemize}
\end{frame}

\begin{frame}
    \frametitle{Ejemplo: Suma de numeros unitarios}
    \begin{itemize}
        \item{Especificar el caso base}
        \item{Especificar los casos recursivos}
    \end{itemize}
\end{frame}

\begin{frame}
    \frametitle{Recursi\'on: Terminaci\'on}
    \begin{itemize}
        \item{Se comienza con una estructura definida inductivamente (como los numeros unitarios)}
        \item{Toda \emph{llamada recursiva} debe realizarse con una \emph{instancia m\'as simple} de dicha estructura}
        \item{Un elemento $\mathcal{A}$ es \emph{m\'as simple} que un elemento $\mathcal{B}$ si
        el elemento $\mathcal{B}$ se puede obtener mediante inducci\'on a partir del elemento $\mathcal{A}$}
        \item{Ejemplo: $\sigma(\sigma(0))$ es m\'as simple que $\sigma(\sigma(\sigma(0)))$}
    \end{itemize}
\end{frame}

\end{document}