%%%%%%%%%%%%%%%%%%%%%%%%%%%%%%%%%%%%%%%%%
% Programming/Coding Assignment
% LaTeX Template
%
% This template has been downloaded from:
% http://www.latextemplates.com
%
% Original author:
% Ted Pavlic (http://www.tedpavlic.com)
%
% Note:
% The \lipsum[#] commands throughout this template generate dummy text
% to fill the template out. These commands should all be removed when 
% writing assignment content.
%
% This template uses a Perl script as an example snippet of code, most other
% languages are also usable. Configure them in the "CODE INCLUSION 
% CONFIGURATION" section.
%
%%%%%%%%%%%%%%%%%%%%%%%%%%%%%%%%%%%%%%%%%

%----------------------------------------------------------------------------------------
%	PACKAGES AND OTHER DOCUMENT CONFIGURATIONS
%----------------------------------------------------------------------------------------

\documentclass{article}

    \usepackage{fancyhdr} % Required for custom headers
    \usepackage{lastpage} % Required to determine the last page for the footer
    \usepackage{extramarks} % Required for headers and footers
    \usepackage[usenames,dvipsnames]{color} % Required for custom colors
    \usepackage{graphicx} % Required to insert images
    \usepackage{listings} % Required for insertion of code
    \usepackage{courier} % Required for the courier font
    \usepackage{multirow}
    \usepackage{hyperref}
    
    
    % Margins
    \topmargin=-0.45in
    \evensidemargin=0in
    \oddsidemargin=0in
    \textwidth=6.5in
    \textheight=9.0in
    \headsep=0.25in
    
    \linespread{1.1} % Line spacing
    
    %----------------------------------------------------------------------------------------
    %	CODE INCLUSION CONFIGURATION
    %----------------------------------------------------------------------------------------
    
    \definecolor{MyDarkGreen}{rgb}{0.0,0.4,0.0} % This is the color used for comments
    \lstloadlanguages{c} % Load Perl syntax for listings, for a list of other languages supported see: ftp://ftp.tex.ac.uk/tex-archive/macros/latex/contrib/listings/listings.pdf
    \lstset{language=[sharp]c, % Use Perl in this example
            frame=single, % Single frame around code
            basicstyle=\small\ttfamily, % Use small true type font
            keywordstyle=[1]\color{Blue}\bf, % Perl functions bold and blue
            keywordstyle=[2]\color{Purple}, % Perl function arguments purple
            keywordstyle=[3]\color{Blue}\underbar, % Custom functions underlined and blue
            identifierstyle=, % Nothing special about identifiers                                         
            commentstyle=\usefont{T1}{pcr}{m}{sl}\color{MyDarkGreen}\small, % Comments small dark green courier font
            stringstyle=\color{Purple}, % Strings are purple
            showstringspaces=false, % Don't put marks in string spaces
            tabsize=5, % 5 spaces per tab
            %
            % Put standard Perl functions not included in the default language here
            morekeywords={rand},
            %
            % Put Perl function parameters here
            morekeywords=[2]{on, off, interp},
            %
            % Put user defined functions here
            morekeywords=[3]{test},
               %
            morecomment=[l][\color{Blue}]{...}, % Line continuation (...) like blue comment
            numbers=left, % Line numbers on left
            firstnumber=1, % Line numbers start with line 1
            numberstyle=\tiny\color{Blue}, % Line numbers are blue and small
            stepnumber=5 % Line numbers go in steps of 5
    }
    
    \newcommand{\horrule}[1]{\rule{\linewidth}{#1}}
    
    % Creates a new command to include a perl script, the first parameter is the filename of the script (without .pl), the second parameter is the caption
    \newcommand{\perlscript}[2]{
    \begin{itemize}
    \item[]\lstinputlisting[caption=#2,label=#1]{#1.cs}
    \end{itemize}
    }
    
    \begin{document}
    
    \begin{tabular}{l l}
     & Universidad del Istmo de Guatemala \\
     & Facultad de Ingenieria \\
     & Ing. en Sistemas \\
     & Informatica 2 \\
     & Prof. Ernesto Rodriguez - \href{mailto:erodriguez@unis.edu.gt}{erodriguez@unis.edu.gt} \\
    \end{tabular}
    \\\\\\
    
    \begin{center}
            \horrule{0.5pt}
            \huge{Hoja de trabajo \#1} \\
            \large{Fecha de entrega: 25 de Enero, 2018 - 11:59pm} \\
            \horrule{1pt}
    \end{center}
    
    \emph{Instrucciones: Realizar cada uno de los ejercicios siguiendo sus respectivas
    instrucciones. El trabajo debe ser entregado a traves de Github, en su repositorio del curso, colocado en una carpeta llamada "Hoja de trabajo 1".
    Al menos que la pregunta indique diferente, todas las respuestas a preguntas escritas deben presentarse en
    un documento formato pdf, el cual haya sido generado mediante Latex. Los ejercicios de programaci\'on deben ser colocados en una carpeta
    llamada ``Programas", la cual debe colocarse dentro de la carpeta correspondiente a esta hoja de trabajo.}
    
    % \perlscript{homework_example}{Sample Perl Script With Highlighting}
    
    \section*{Iniciaci\'on}
    \begin{enumerate}
            \item{Si no la tiene, crear una cuenta en \href{https://github.com/}{Github}.}
            \item{Crear un repositorio\cite{GitRepo} para los deberes correspondientes a este curso en Github.}
            \item{Enviar su usuario de Github y link de su repositorio al profesor
            \href{mailto:erodriguez@unis.edu.gt}{erodriguez@unis.edu.gt}.}
            \item{Tomar en cuenta que estea hoja
            (y las hojas restantes del curos) deben ser entregadas mediante un commit
            en el repositorio creado en el articulo 2 de Github. En la fecha y hora de entrega,
            el profesor copiara el estado acutal del repositorio y calificara ese material. Por
            favor asegurarse que todos los cambios esten \emph{empujados} (\texttt{git push}) en el
            repositorio antes de la fecha de entrega. Esto se puede revisar accediendo el sitio
            Github y ver que todos los archivos necesarios esten presentes en el repositorio.}
            \item{El ``Ejercicio \#1'' y ``Ejercicio \#4'' le ayudaran a publicar correctamente
            el trabajo en Github. Si aun tiene dudas, por favor contactar al profesor.}
            \item{Para hacer esta hoja de trabajo, necesitara Latex\cite{LatexPage}. Para trabajar
            con archivos Latex (``*.tex'') se recomienda utilizar un editor. La \href{https://marketplace.visualstudio.com/items?itemName=James-Yu.latex-workshop}{extension de
            Latex} para \emph{Visual Studio Code} es excelente, pero t\'ambien se puede utilizar
            un programa como \href{http://www.xm1math.net/texmaker/}{Texmaker}. Asegurese
            de configurar Latex de manera que pueda producir archivos \emph{PDF}.}
            \item{El Wiki de Latex\cite{Latex} es un buen lugar para consultar el formato.}
    \end{enumerate}
    
    \section*{Ejercicio \#1 (5\%)}
    \begin{enumerate}
            \item{
                    Si no lo ha hecho, crear un repositorio en Github para alojar
                    las hojas de trabajo.}
            \item{
                    Luego de haber creado un repositorio, debe clonar el repositorio
                    a una carpeta local mediante el comando \texttt{git clone [url]} el cual
                    puede ser ejecutado desde la terminal (Linux y Mac) o desde la consola
                    \emph{Git Bash} (en Windows). Para mayor
                    facilidad, utilizar un url tipo \emph{https}. Sin embargo para mayor
                    seguridad y conveniencia se recomienda usar \emph{ssh}\cite{GithubSsh}.
                    Como ejemplo, este comando clonaria el repositorio de este curso: \\
                    \texttt{git clone https://github.com/netogallo/Informatica-2.git}
            }
            \item{
                    Luego de haber clonado el repositorio, se creara una carpeta
                    con el nombre del repositorio dentro de la carpeta donde se
                    haya ejecutado el comando anterior.
            }
    \end{enumerate}
    
    \section*{Ejercicio \#2 (80\%)}
    El proyecto ``QueHaceres" constara de dos objetos reales. Un
    ``Que hacer" el cual corresponde a un deber o obligaci\'on que
    necesitamos cumplir, la cual no tiene \emph{ninguna} fecha limite.
    Un ``Que hacer" puede estar completado o en progreso.
    
    El segundo objeto es una lista de ``Que haceres", la cual permite agregar
    nuevos ``Que haceres", visualizar todos los ``Que haceres" disponibles y
    marcar un ``Que hacer" como terminado.
    
    \begin{enumerate}
            \item{
                    Describir al objeto concreto ``Que Hacer" mediante un objeto
                    abstracto. Listar todos los attributos que tendra el objeto
                    abstracto y el tipo de cada atributo.}
            \item{
                    Describir al objeto concreto ``Que haceres" mediante un objeto
                    abstracto. Listar todos los atributos que tendra el objeto abstracto
                    y el tipo de cada atributo.
            }
    \end{enumerate}
    
    \section*{Ejercicio \#3 (10\%)}
    Crear un archivo \emph{.gitignore} en la carpeta principal de este repositorio.
    Asegurarse que git excluya todos los archivos menos los terminados en ``.tex'' y ``.pdf''.
    
    \section*{Ejercicio \#4 (5\%)}
    Luego de haber terminado los ejercicios anteriores, su trabajo debe
    ser publicado a github. Para publicar a Github utilizando Git se procede
    de la siguiente manera:
    \begin{enumerate}
            \item {
                    El comando \texttt{git status} listara todos los
                    cambios pendientes por agregar al repositorio. Si no
                    ha agregado ningun archivo al repositorio, listara todas
                    las carpetas y el archivo \emph{.gitignore} dentro de la
                    carpeta del proyecto.
            }
            \item{
                    El comando \texttt{git add [archivos]} permite agregar
                    archivos a la siguiente revision de git. Como atajo,
                    puede ejecutar \texttt{git add .} para agregar todos
                    los cambios que hay en el repositorio.
            }
            \item {
                    El comando \texttt{git status} ahora indicara los archivos
                    nuevos que hay en el repositorio. Si su archivo \emph{.gitignore}
                    esta formateado correctamente, solo deberian agregarse:
                    \begin{itemize}
                            \item{El archivo \emph{.gitignore}}
                            \item{El archivo \emph{pdf} correspondiente a esta hoja de trabajo}
                            \item{El archivo \emph{tex} correspondiente a esta hoja de trabajo}
                            \item{Un posible archivo o carpeta de configuraci\'on del editor.
                            El caso de ``Visual Studio Code'' sera una carpeta llamada \emph{.vscode}.}
                    \end{itemize}
            }
            \item{
                    Si el paso anterior muesta archivos extra (``*.aux'', ``*.log'', ``*.out'', ect...),
                    debera componer su archivo \emph{.gitignore}. Para hacerlo, primero
                    debe ejecutar el comando \texttt{git reset HEAD .}, el cual quitara
                    todos los archivos de la lista de cambios. Luego modifique su archivo \emph{.gitignore}
                    adecuadamente y vuelba a ejecutar \texttt{git add .}. Repetir hasta
                    que la lista de archivos nuevos corresponda a la indicada en el paso \#3.
            }
    \end{enumerate}
    
    
    \bibliography{../../Referencias/referencias}
    \bibliographystyle{plain}
    
    \end{document}