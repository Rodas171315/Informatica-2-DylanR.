\documentclass[10pt,a4paper]{article}

\usepackage{fancyhdr}
\usepackage{lastpage}
\usepackage{extramarks}
\usepackage[utf8]{inputenc}
\usepackage[spanish]{babel}
\usepackage{amsmath}
\usepackage{amsfonts}
\usepackage{amssymb}
\usepackage{graphicx}
\usepackage[usenames,dvipsnames]{color}
\usepackage{listings}
\usepackage{courier}
\usepackage{multirow}
\usepackage{hyperref}

\usepackage[left=2cm,right=2cm,top=2cm,bottom=2cm]{geometry}

\author{Dylan Rodas}
\title{HojaDeTrabajo1}

\newcommand{\horrule}[1]{\rule{\linewidth}{#1}}

\begin{document}

	\begin{tabular}{l l}
     \multirow{5}{*}{\includegraphics[width=2cm]{D:/Documentos/UNIS/Logo_UNIS.png}} & Universidad del Istmo de Guatemala \\
     & Facultad de Ingenieria \\
     & Ing. en Sistemas \\
     & Informatica 2 \\
     & Dylan Gabriel Rodas Samayoa - \href{mailto:rodas171315@unis.edu.gt}{rodas171315@unis.edu.gt} \\
    \end{tabular}
	\\\    
	
    \begin{center}
        \horrule{0.5pt}
        \huge{Hoja de Trabajo \#1} \\
        \large{25 de Enero, 2018} \\
        \horrule{1pt}
	\end{center}

	\begin{center}
	\section*{Resolución Ejercicio \#2}
	\section*{Proyecto ``QueHaceres"}
	\end{center}

	\section*{Objeto ``QueHacer"}
	
Deber o obligación que necesitamos cumplir, la cual no tiene ninguna fecha limite. Puede estar completado o en progreso. Describe cualquier tarea que pueda haber. Entre sus propiedades podemos encontrar:

	\begin{enumerate}
\item{Constantes: Valores constantes asociados. Tipo: int, uint, char.}
\item{Campos: Variables.}
\item{Métodos: Cálculos y acciones que pueden realizarse.}
\item{Propiedades: Acciones asociadas a la lectura y escritura de propiedades. Tipo: get, set.}
\item{Indizadores: Acciones asociadas a la indexación de instancias como una matriz.}
\item{Eventos: Notificaciones que puede generar.}
\item{Operadores: Conversiones y operadores de expresión admitidos. Tipo: ==, !=, en enum.}
\item{Constructores: Acciones necesarias para inicializar instancias. Tipo: new.}
\item{Finalizadores: Acciones que deben realizarse antes de que las instancias se descarten de forma permanente.}
\item{Accesibilidad: Controla las regiones del texto del programa que pueden tener acceso al miembro. Tipo: public, protected, private, internal.}
    \end{enumerate}
    
	\section*{Objeto ``Quehaceres"}
	
Es una lista de ``Que haceres", la cual permite agregar nuevos ``Que haceres", visualizar todos los disponibles ``Que haceres", y marcar como terminado un ``Que hacer".

	\begin{enumerate}
\item{Para ejecutar cálculos: int Sumar (ejecuta una adición entre dos elementos), int Restar (ejecuta una sustracción entre dos elementos), int Multiplicar (ejecuta un producto entre dos elementos), int Dividir (ejecuta un cociente entre dos elementos). Todas las propiedades representan números enteros.}
\item{Para determinar el tiempo: int Segundo, int Minuto, int Hora, int Día. Las propiedades se emplean para marcar el tiempo dedicado.}
\item{Para definir una persona: string Nombres, string Apellidos, string Sexo, int Edad, string Nacionalidad. Con estas propiedades se establece un sujeto.}
\item{Para mandar un mensaje: Usuario, Escribir, Enviar, Recibir. Propiedades para generar una interacción entre dos objetos.}
\item{Para realizar listas: Tareas, Agregar, Visualizar, Marcar, Ordenar, Eliminar. Propiedades de instancias que interactúan entre ellas para crear nuevas tareas.}
    \end{enumerate}
	
\end{document}