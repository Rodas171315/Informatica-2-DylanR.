\documentclass[10pt,a4paper]{article}

\usepackage{fancyhdr}
\usepackage{lastpage}
\usepackage{extramarks}
\usepackage[utf8]{inputenc}
\usepackage[spanish]{babel}
\usepackage{amsmath}
\usepackage{amsfonts}
\usepackage{amssymb}
\usepackage{graphicx}
\usepackage[usenames,dvipsnames]{color}
\usepackage{listings}
\usepackage{courier}
\usepackage{multirow}
\usepackage{hyperref}

\usepackage[left=2cm,right=2cm,top=2cm,bottom=2cm]{geometry}

\author{Dylan Rodas}
\title{HojaDeTrabajo1}

\newcommand{\horrule}[1]{\rule{\linewidth}{#1}}

\begin{document}

	\begin{tabular}{l l}
     \multirow{5}{*}{\includegraphics[width=2cm]{D:/Documentos/UNIS/Logo_UNIS.png}} & Universidad del Istmo de Guatemala \\
     & Facultad de Ingenieria \\
     & Ing. en Sistemas \\
     & Informatica 2 \\
     & Dylan Gabriel Rodas Samayoa - \href{mailto:rodas171315@unis.edu.gt}{rodas171315@unis.edu.gt} \\
    \end{tabular}
	\\\    
	
    \begin{center}
        \horrule{0.5pt}
        \huge{Hoja de Trabajo \#1} \\
        \large{25 de Enero, 2018} \\
        \horrule{1pt}
	\end{center}

	\begin{center}
	\section*{Ejercicio \#2}
	\end{center}
	
	\section*{``QueHacer"}
	
	\begin{enumerate}
            \item{Copiar un texto.}
            \item{Lavar al perro.}
            \item{Cocinar Sopa Maruchan.}
            \item{Encender el televisor.}
            \item{Comer.}
            \item{Comprar churros de la Feria.}
    \end{enumerate}
    
	\section*{``Quehaceres"}
	
	\begin{enumerate}
            \item{Seleccionar el texto de origen, fotocopiar el texto 	y pegar el texto en el destino.}
            \item{Atar al perro, echarle agua al perro, aplicarle jabón al perro, masajear al perro, echarle agua al perro, desatar al perro.}
            \item{Seleccionar sabor de sopa, quitar envolvente de plástico de sopa, abrir tapa de sopa, echar agua a la sopa, abrir microondas, insertar sopa en microondas, cerrar microondas, determinar tiempo de microondas, empezar tiempo de microondas, esperar, abrir microondas, retirar sopa, cerrar microondas, degustar.}
            \item{Comprar televisor, conectar televisor a tomacorrientes, buscar control de televisor, buscar boton de encendido en control, oprimir boton de encendido, visualizar televisor.}
            \item{Buscar plato limpio, colocar comida en plato limpio, buscar cubierto limpio, comer, generar plato sucio, generar cubierto sucio, depositar plato y cubierto sucios en lavaplatos.}
            \item{Buscar feria, entrar a feria, buscar carro de churros, moverse a carro de churros, robar churros, buscar salida de feria, correr a salida de feria, degustar churros.}
    \end{enumerate}
	
\end{document}