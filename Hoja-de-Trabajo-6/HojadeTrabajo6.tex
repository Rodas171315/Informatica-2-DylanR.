\documentclass[10pt,a4paper]{article}

    \usepackage{fancyhdr}
    \usepackage{lastpage}
    \usepackage{extramarks}
    \usepackage[utf8]{inputenc}
    \usepackage[spanish]{babel}
    \usepackage{amsmath}
    \usepackage{amsfonts}
    \usepackage{amssymb}
    \usepackage{graphicx}
    \usepackage[usenames,dvipsnames]{color}
    \usepackage{listings}
    \usepackage{courier}
    \usepackage{multirow}
    \usepackage{hyperref}
    
    \usepackage[left=2cm,right=2cm,top=2cm,bottom=2cm]{geometry}
    
    \author{Dylan Rodas}
    \title{HojaDeTrabajo6}
    
    \newcommand{\horrule}[1]{\rule{\linewidth}{#1}}
    
    \begin{document}
    
        \begin{tabular}{l l}
         \multirow{5}{*}{\includegraphics[width=2cm]{D:/Documentos/UNIS/Logo_UNIS.png}} & Universidad del Istmo de Guatemala \\
         & Facultad de Ingenieria \\
         & Ing. en Sistemas \\
         & Informatica 2 \\
         & Dylan Gabriel Rodas Samayoa - \href{mailto:rodas171315@unis.edu.gt}{rodas171315@unis.edu.gt} \\
        \end{tabular}
        \\\    
        
        \begin{center}
            \horrule{0.5pt}
            \huge{Hoja de Trabajo \#6} \\
            \large{25 de Marzo, 2018} \\
            \horrule{1pt}
        \end{center}
    
        \begin{center}
        \section*{Resoluci\'on Ejercicio \#1}
        \section*{Definici\'on Recursiva}
        \end{center}

        \begin{itemize}
            \item\textit{Dado un numero n que pertenece a los numero naturales unitarios y que este sea un n sucesor de cero ($\sigma n(0) = n$).}
            \item\textit{Y cero (0) el primer numero de los numeros naturales unitarios.}
        \end{itemize}
    
        \section*{Suma de dos n\'umeros}
        
        Caso base: 
        \begin{center}
        $n + 0 = n$\\
        $\sigma(n) + m = \sigma(n+m)$
        \end{center}
        Caso inductivo:
        \begin{center}
        $\sigma(\sigma(\sigma(0))) + \sigma(\sigma(0))$\\
        $\sigma(\sigma(\sigma(0)) +\sigma(\sigma(0))) $\\
        $\sigma(\sigma(\sigma(0) +\sigma(\sigma(0)))) $\\
        $\sigma(\sigma(\sigma(0 +\sigma(\sigma(0))))) $\\
        $\sigma(\sigma(\sigma(\sigma(\sigma(0)))))$
        \end{center}
        \emph{De manera que:\\
        $\sigma(n) = a$, $m = b$, $\sigma(n+m)= c$}
    
        \section*{Multiplicaci\'on de dos n\'umeros}
        
        Caso base: 
        \begin{center}
        $n * 0 =0$\\
        $\sigma(n) * m = \sigma((n)*m) +m$\\
        $\sigma(0) * n = \sigma(0 + n)$\\
        \end{center}
        Caso inductivo:
        \begin{center}
		$(\sigma(0) * \sigma(\sigma(0))$\\
		$\sigma(0) + \sigma(0) + \sigma(\sigma(0))veces...+\sigma(0)$\\
		$\sigma(0) + [\sigma(0) + \sigma(\sigma(0))veces...+\sigma(0)] $\\
		$\sigma(0)+ [\sigma(0) * (\sigma(0)] $\\
		$\sigma(\sigma(0))$
        \end{center}
        \emph{De manera que:\\
        $\sigma(n) = a$, $m = b$, $\sigma(n*m) = c$}
            
        \section*{Mayor que para n\'umeros unitarios}
        
        Caso base: 
        \begin{itemize}
        \item$\sigma(0) > 0$
        \item$\sigma(\sigma(n))>n$
        \end{itemize}
        Caso inductivo:
        \begin{itemize}
        \item$\sigma(\sigma(0)) > \sigma(0)$\\
        $\sigma(0) > 0$\\
        \item$\sigma(\sigma(n)) > n$\\
        $\sigma(\sigma(\sigma(n))) > \sigma(n)$\\
        $\sigma(\sigma(n))>n$
        \end{itemize}
        
        \begin{center}
        \section*{Resoluci\'on Ejercicio \#2}
        \section*{Propiedades con Inducci\'on}
        \end{center}
        
        \subsection{Demostracion 1}
        \begin{center}
        $n+0=n$:\\
        $\sigma(n) + 0 = \sigma(n)$\\
        $\sigma (n + 0) = \sigma(n)$\\
        $\sigma(n) = \sigma(n)$
        \end{center}
        \subsection{Demostracion 2}
        \begin{center}
        $n+m = m+n$:\\
        $\sigma(\sigma(0)) + \sigma(\sigma(\sigma(0)))$ = $\sigma(\sigma(\sigma(0))) + \sigma(\sigma(0))$\\
        $\sigma(\sigma(0) + \sigma(\sigma(\sigma(0))))$ = $\sigma(\sigma(\sigma(0)) + \sigma(\sigma(0)))$\\
        $\sigma(\sigma(0 + \sigma(\sigma(\sigma(0)))))$ = $\sigma(\sigma(\sigma(0) + \sigma(\sigma(0))))$\\
        $\sigma(\sigma(\sigma(\sigma(\sigma(0)))))$ = $\sigma(\sigma(\sigma(0 + \sigma(\sigma(0)))))$\\
        $\sigma(\sigma(\sigma(\sigma(\sigma(0)))))$ = $\sigma(\sigma(\sigma(\sigma(\sigma(0)))))$
        \end{center}
        \subsection{Demostracion 3}
        \begin{center}
        $n * \sigma(\sigma(0)) = n + n$:\\
        $\sigma(n) * \sigma(\sigma(0)) = \sigma(n) + \sigma(n)$\\
        $\sigma(\sigma(0) + n) =  \sigma(n) + \sigma(n)$\\
        $\sigma(\sigma(0+n)) = \sigma(n + \sigma(n))$\\
        $\sigma(\sigma(n)) = \sigma(\sigma(n))$\\
        $\sigma(n) + \sigma(n) = \sigma(n) + \sigma(n)$
        \end{center}
        
    \end{document}