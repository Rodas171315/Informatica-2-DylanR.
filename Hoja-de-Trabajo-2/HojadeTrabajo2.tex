\documentclass[10pt,a4paper]{article}

\usepackage{fancyhdr}
\usepackage{lastpage}
\usepackage{extramarks}
\usepackage[spanish]{babel}
\usepackage{amsmath}
\usepackage{amsfonts}
\usepackage{amssymb}
\usepackage{graphicx}
\usepackage[usenames,dvipsnames]{color}
\usepackage{listings}
\usepackage{courier}
\usepackage{multirow}
\usepackage{hyperref}

\usepackage[left=2cm,right=2cm,top=2cm,bottom=2cm]{geometry}

\author{Dylan Rodas}
\title{Tarea Filo 1}

\newcommand{\horrule}[1]{\rule{\linewidth}{#1}}

\begin{document}

	\begin{tabular}{l l}
     \multirow{5}{*}{\includegraphics[width=2cm]{D:/Documentos/UNIS/Logo_UNIS.png}} & Universidad del Istmo de Guatemala \\
     & Facultad de Ingenier\'ia \\
     & Ing. en Sistemas \\
     & Inform\'atica 2 \\
     & Dylan Gabriel Rodas Samayoa - \href{mailto:rodas171315@unis.edu.gt}{rodas171315@unis.edu.gt} \\
    \end{tabular}
	\\\    
	
    \begin{center}
        \horrule{0.5pt}
        \huge{Hoja de Trabajo \#2} \\
        \large{1 de Febrero, 2018} \\
        \horrule{1pt}
	\end{center}
	\
	\begin{center}
	\section*{Ejercicio \#5: Confusi\'on}
	\end{center}
	\
	\section*{¿A qu\'e se debe ese extraño resultado?}

\item{Como primera observaci\'on, en la cuarta l\'inea, hay un punto donde deber\'ia haber una coma, as\'i que, el c\'odigo no va a compilar.}
\item{Como segunda observaci\'on, al usar un operador condicional (? :) es un operador ternario (toma tres operandos). El operador condicional funciona del modo siguiente:}

	\begin{itemize}
\item{El primer operando se convierte implícitamente a bool. Se evalúa y todos los efectos secundarios se completan antes de continuar.}
\item{Si el primer operando se evalúa como true (1), se evalúa el segundo operando.}
\item{Si el primer operando se evalúa como false (0), se evalúa el tercer operando.}
    \end{itemize}

\end{document}